\documentclass[]{article}

%Language settings
\usepackage[spanish]{babel}

%Paper size and margins
\usepackage[letterpaper,top=2.54cm,bottom=2.54cm,left=2.54cm,right=2.54cm,marginparwidth=1.75cm]{geometry}

%Other packages
\usepackage{amsmath}
\usepackage{graphicx}
\usepackage[colorlinks=true, allcolors=blue]{hyperref}

\title{1. Divisor de voltaje}
\author{Flores Tun, Jorge David; López Gómez, Wilberth Eduardo; Sánchez Soberanis, Felipe}

\begin{document}
\maketitle

\begin{abstract}
\end{abstract}

\section{Introducción}

Mediante el uso de los teoremas de máxima transferencia de potencia y divisor de voltaje, se presentará un circuito que cuya función es mantener encendido un foco ubicado en un 
arreglo de divisor de voltaje procurando que el voltaje de caída en ese nodo sea menor a 2 V cuando éste se encienda. Para esta práctica, se empleó el software de SparkFun para 
simular el circuito antes de realizarlo, al igual que se empleó un número "n" de resistencias de un mismo valor tanto para la simulación como para el circuito físico.

\section{Marco teórico}

\subsection{Máxima Transferencia de Potencia}

El teorema de transferencia de potencia máxima se da cuando se disipa la cantidad máxima de energía por medio de una resistencia de carga que es igual a la resistencia de Thevenin/Norton
del nodo de orígen, por lo que da como respuesta una potencia disipada que será menor que la máxima. 

De igual manera, es importante recalcar que la potencia máxima no es equivalente a la máxima eficiencia en el sistema.

\subsection{Divisor de voltaje}

El divisor de voltaje es la relación que tienen los nodos de dos resistencias en serie de modo que se relacionan su voltaje de salida $v_out$ con el de entrada $v_in$.

En un circuito, cuando se tiene un arreglo de resistencias, éstas se rigen por la ley de Ohm, la cual establece que:

\begin{equation}{
    V=IR
    }



\section{Instrucciones}

Mediante un circuito constituido por una fuente, dos resistencias equivalentes y
la teoría de la division de voltaje y la máxima transferencia de energía, se
pretende diseñar un arreglo de $n$ resistencias para realizar el encendido de
un foco, el cual es generado mediante un encendido del cual no sufra una caida
de voltaje menor a 2 V.

\section{Materiales}

\begin{itemize}
    \item Fuente de voltaje que pueda brindar una alimentación de 20 V.
    \item Resistencias (varias).
    \item Foco
    \item Switch
    \end{itemize}


\section{Desarrollo}


Mediante la fórmula de divisor de voltaje, se obtuvo el número de resistencias
necesarias para $R_1$ y $R_2$.

\section{Resultados}

Para la resistencia $R_1$ se obtuvo que requiere de 14 resistencias de 100
$\Omega$ para obtener una resistencia equivalente $R_1 = 8 \Omega$ y 14 resistecias
de $150 \Omega$ para obtener una resistencia de $R_2 = 12 \Omega$.


\section{Conclusiones}

\bibliographystyle{alpha}


\end{document}
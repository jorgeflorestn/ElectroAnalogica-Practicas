\documentclass[]{article}

%Language settings
\usepackage[spanish]{babel}

%Paper size and margins
\usepackage[letterpaper,top=2.54cm,bottom=2.54cm,left=2.54cm,right=2.54cm,marginparwidth=1.75cm]{geometry}

%Other packages
\usepackage{amsmath}
\usepackage{graphicx}
\usepackage[colorlinks=true, allcolors=blue]{hyperref}

\title{1. Divisor de voltaje}
\author{Flores Tun, Jorge David; López Gómez, Wilberth Eduardo; Sánchez Soberanis, Felipe}

\begin{document}
\maketitle

\begin{abstract}
    En este reporte se presentarán los resultados de la práctica
\end{abstract}

\section{Introducción}



\section{Marco teórico}



\section{Instrucciones}

Mediante un circuito constituido por una fuente, dos resistencias equivalentes y
la teoría de la division de voltaje y la máxima transferencia de energía, se
pretende diseñar un arreglo de $n$ resistencias para realizar el encendido de
un foco, el cual es generado mediante un encendido del cual no sufra una caida
de voltaje menor a 2 V.

\section{Materiales}

\begin{itemize}
    \item Fuente de voltaje que pueda brindar una alimentación de 20 V.
    \item Resistencias (varias).
    \item Foco
    \item Switch
    \end{itemize}


\section{Desarrollo}


Mediante la fórmula de divisor de voltaje, se obtuvo el número de resistencias
necesarias para $R_1$ y $R_2$.

\section{Resultados}

Para la resistencia $R_1$ se obtuvo que requiere de 14 resistencias de 100
$\Omega$ para obtener una resistencia equivalente $R_1 = 8 \Omega$ y 14 resistecias
de $150 \Omega$ para obtener una resistencia de $R_2 = 12 \Omega$.


\section{Conclusiones}

\bibliographystyle{alpha}


\end{document}